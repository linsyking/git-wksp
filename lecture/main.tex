\documentclass[aspectratio=169]{beamer}
\usepackage{listings}
\usepackage{graphicx}
\usepackage{hyperref}
\usepackage{tikz}
\usepackage{minted}
\definecolor{UMBlue}{RGB}{5,32,103}
\definecolor{UMYellow}{RGB}{255,216,0}
\setminted{
    breaklines,
    fontsize=\footnotesize,
    xleftmargin=0em
}
\setbeamertemplate{headline}{
  \leavevmode%
  \begin{minipage}{0.75\paperwidth}
  \vspace{1.2ex}\hspace{-0.245\paperwidth}
  \resizebox{\paperwidth}{3ex}{
    \tikz{
      \fill [color=UMBlue] (0,0) rectangle (10, 0.13);
      \fill [color=UMYellow] (0,-0.1) rectangle (10, -0.18);
    }
  }
  \begin{beamercolorbox}[wd=\linewidth,ht=2.5ex,dp=1.125ex]{section}
    \insertsubsectionnavigationhorizontal{\linewidth}{}{Slide \insertpagenumber}
  \end{beamercolorbox}
  \end{minipage}
  \begin{minipage}{0.23\paperwidth}
  \hspace{0.2em}
  \includegraphics[width=0.9\textwidth]{techji.jpg}
  \end{minipage}
}

\setbeamercolor{title}{fg=UMBlue}
\setbeamercolor{frametitle}{fg=UMBlue}
\setbeamercolor{structure}{fg=UMBlue}

\setbeamertemplate{navigation symbols}{}

\hypersetup{
    colorlinks=true,
    linkcolor=UMBlue,
    filecolor=magenta,
    urlcolor=UMBlue
}

\urlstyle{same}

\title{Git: Practice and Applications}
\author{Yiming Xiang}
\institute{UM-SJTU Joint Institute}
\date{\today}

\AtBeginSection[]{
    \begin{frame}{Table of Contents}
        % \small
        % \begin{multicols}{2}
        \tableofcontents[currentsection]
        % \end{multicols}
    \end{frame}
}

\begin{document}
\frame{\titlepage}

\begin{frame}{Before we start...}
    We assume you:
    \pause
    \begin{itemize}[<+->]
        \item have used Github or similar services before
        \item have used Git to manage your projects
        \item know basic git commands: \texttt{git add}, \texttt{git commit}, \texttt{git push}, \texttt{git pull}, \texttt{git clone}, \texttt{git checkout}, \texttt{git merge}
        \item want to know more about git
    \end{itemize}
\end{frame}

\begin{frame}{Before we start...}
    What you should expect in this workshop:
    \begin{itemize}
        \item A moderate understanding of Git branching
        \item lazygit
        \item Some advanced Git commands
        \item Git LFS \& Git Annex
    \end{itemize}

    \pause

    What you should not expect in this workshop:
    \begin{itemize}
        \item Why Git is better than SVN
        \item Why should I use Github
        \item How to use Github
        \item How to remember Git commands
    \end{itemize}
\end{frame}

\section{A Crash Course in Git Branching}

\begin{frame}{A Simple Model of Git Commits and Branches}
    \begin{itemize}
        \item A \emph{commit} is a \textbf{snapshot} of your project, along with some \textbf{meta info}, including commit id, author, etc, and also \textbf{the pointer to the previous commit}, called the \emph{parent} of the commit. Note that Git only stores \textbf{pointers} to files in commits, not the files themselves.
        \item A \emph{branch} is a \textbf{movable} pointer to a commit. Every time you commit, the branch pointer moves to the new commit.
        \item A \emph{tag} is a \textbf{fixed} pointer to a commit.
        \item \emph{HEAD} is a special pointer that \textbf{points to the current branch}. Checking out a branch is simply moving the HEAD pointer to another branch. You can also \emph{detach} HEAD to point to a specific commit.
    \end{itemize}
\end{frame}

\begin{frame}{Commit Calculus}
    First, every commit has a unique ID (SHA-hash), often presented as a hexadecimal string. Usually the first 7 characters are enough to identify a commit.

    \smallskip
    \pause

    Second, please remember that \textbf{commits are immutable}. Once you make a commit, Git will never change it or delete it. Moreover, remember that \textbf{commit doesn't store changes}. Commit is a snapshot. Therefore, changes are calculated by comparing two commits.

    \smallskip
    \pause

    Sometimes we need to find the parent of a commit or dump the differences between two commits. Git provides the following conventions:
    \begin{itemize}
        \item \~{}: (linear) parent commit. \texttt{HEAD\~{}} is the parent of the current commit. \texttt{HEAD\~{}2} is the grandparent of the current commit.
        \item \^{}: (non-linear) first parent commit. \texttt{HEAD\^{}2} is the second parent of the current commit. You can combine this with \~{} to get the second parent of the parent of the current commit.
    \end{itemize}
\end{frame}

\begin{frame}[fragile]{Commit Calculus}
    An \href{https://mirrors.edge.kernel.org/pub/software/scm/git/docs/git-rev-parse.html}{illustration} by \emph{Jon Loeliger} of the commit calculus:
    \bigskip
    \begin{columns}
        \begin{column}{0.2\textwidth}
            \begin{minted}{bash}
G   H   I   J
\ /     \ /
 D   E   F
  \  |  / \
   \ | /   |
    \|/    |
     B     C
      \   /
       \ /
        A
            \end{minted}
        \end{column}
        \begin{column}{0.5\textwidth}
            \begin{minted}{bash}
A =      = A^0
B = A^   = A^1     = A~1
C = A^2
D = A^^  = A^1^1   = A~2
E = B^2  = A^^2
F = B^3  = A^^3
G = A^^^ = A^1^1^1 = A~3
H = D^2  = B^^2    = A^^^2  = A~2^2
I = F^   = B^3^    = A^^3^
J = F^2  = B^3^2   = A^^3^2
            \end{minted}
        \end{column}
    \end{columns}
\end{frame}

\begin{frame}{Three Areas}
    Here ``area'' simply means a ``collection of files''.
    Git manages three areas:
    \begin{itemize}
        \item HEAD.
        \item Index. Proposed next commit. Also known as the \textbf{Staging Area}. When you \texttt{git add} a file, it is added to the index.
        \item Working Directory. This is often what you see outside the \texttt{.git} folder. It makes users easier to edit files. You can think of the working directory as a sandbox where you can edit files without affecting any commits.
    \end{itemize}
\end{frame}

\begin{frame}{Workflow}
    \begin{figure}
        \centering
        \includegraphics[width=0.7\textwidth]{workflow.png}
        \caption{A typical Git workflow. \emph{Pro Git}.}
    \end{figure}
\end{frame}

\begin{frame}{References}
    \begin{itemize}
        \item Pro Git. \href{https://git-scm.com/book/en/v2}{Git - Book}
        \item Stack Overflow. \href{https://stackoverflow.com/questions/2221658/whats-the-difference-between-head-and-head-in-git}{What's the difference between head and head in Git?}
        \item TheCW. \emph{Bilibili}. \href{https://www.bilibili.com/video/BV1gV411k7fC}{BV1gV411k7fC}
    \end{itemize}
\end{frame}

\begin{frame}
    \Huge{\textcolor{UMBlue}{Thanks for listening!}}
\end{frame}

\end{document}
